\documentclass{bmstu}

\usepackage{biblatex}
\usepackage{array}
\usepackage{amsmath}

\addbibresource{inc/biblio/sources.bib}

\begin{document}

\makereporttitle
    {Информатика и системы управления}
    {Программное обеспечение ЭВМ и информационные технологии}
    {лабораторным работам №~1--5}
    {Экономика программной инженерии}
    {Планирование программного проекта в Microsoft Project}
    {}
    {Марченко~В./ИУ7-83Б}
    {Барышникова~М.~Ю.}

{\centering \maketableofcontents}

\chapter{Лабораторная работа №~1}

\section{Тема и цель}

Тема: планирование программного проекта в Microsoft Project --- настройка рабочей среды и создание нового проекта.

Цель: освоить возможности программы Microsoft Project для планирования проекта по разработке программного обеспечения.

\section{Тренировочное задание (вариант №~1)}

В ходе выполнения тренировочного задания было осуществлено планирование проекта с учетом определенных связей между задачами. 
Сначала была настроена рабочая среда проекта. 
Единица измерения длительности выполения задач --- дни, 8-часовой рабочий день с 9:00 до 18:00, 5 рабочих дней в неделю, 40 часов в неделю. 
Были учтены праздничные дни --- 8 Марта, День Победы и День международной солидарности трудящихся. 
Дата начала проекта --- 1 марта 2024 г., дата окончания --- 8 мая 2024 г. 
На рисунке~\ref{img:screenshot-01} показана диаграмма Ганта с включенной суммарной задачей.
    
\includeimage
    {screenshot-01}
    {f}
    {H}
    {1\textwidth}
    {Диаграмма Ганта для тренировочного задания №~1}
    
По умолчанию при планировании проекта Microsoft Project использует фиксированный объем ресурсов в качестве типа задач.
    
\section{Основное задание}

Содержание проекта. 
Команда разработчиков из 16-и человек занимается созданием карты города на основе собственного модуля отображения. 
Проект должен быть завершен в течение 6-и месяцев. 
Бюджет проекта: 50 000 рублей.

\subsection{Задание №~1}

В ходе выполнения первого задания была осуществлена настройка рабочей среды проекта с помощью раздела <<Расписание>> окна <<Параметры Project>>. 
На рисунке~\ref{img:screenshot-02} показаны параметры планирования.

Единица измерения длительности выполнения задач --- недели, 8-часовой рабочий день с 9:00 до 18:00, 5 рабочих дней в неделю, 40 часов в неделю. 
Дата начала проекта --- 1 марта 2024 г.
    
\includeimage
    {screenshot-02}
    {f}
    {H}
    {1\textwidth}
    {Параметры планирования для основного задания}
    
Учтены праздничные дни в календаре вплоть до августа 2024 г. 
На экран выведена суммарная задачу проекта и заполнена вкладка <<Заметки>> информацией об основных параметрах проекта.

\subsection{Задание №~2}

В ходе выполнения второго задания был создан список задач. 
На рисунках~\ref{img:screenshot-03-01}--\ref{img:screenshot-03-02} показан список задач.
    
\includeimage
    {screenshot-03-01}
    {f}
    {H}
    {1\textwidth}
    {Список задач --- часть 1}
    
\includeimage
    {screenshot-03-02}
    {f}
    {H}
    {1\textwidth}
    {Список задач --- часть 2}
    
Задачи 1 и 27 являются задачами вехами, поэтому они имеют нулевую
продолжительность. 
Задачи 2, 3, 8, 12, 19 и 22 в задании №~3 будут преобразованы в фазы проекта, поэтому их длительность условная.

\subsection{Задание №~3}

В ходе выполнения третьего задания список задач был структурирован с помощью кнопки <<Понизить уровень задачи>> на вкладке <<Задача>>. 
На рисунках~\ref{img:screenshot-04-01}--\ref{img:screenshot-04-02} показан структурированный список задач.
    
\includeimage
    {screenshot-04-01}
    {f}
    {H}
    {1\textwidth}
    {Структурированный список задач --- часть 1}
    
\includeimage
    {screenshot-04-02}
    {f}
    {H}
    {1\textwidth}
    {Структурированный список задач --- часть 2}

\subsection{Задание №~4}

В ходе выполнения четвертого задания были установлены связи разных типов между задачами. 
На рисунках~\ref{img:screenshot-05-01}--\ref{img:screenshot-05-02} показан результат выполнения основного задания --- план проекта по разработке программного обеспечения.
    
\includeimage
    {screenshot-05-01}
    {f}
    {H}
    {1\textwidth}
    {План проекта по разработке программного обеспечения --- часть 1}
    
\includeimage
    {screenshot-05-02}
    {f}
    {H}
    {1\textwidth}
    {План проекта по разработке программного обеспечения --- часть 2}
    
\section{Вывод}

По ТЗ проект должен был выполниться за 6 месяцев, но при планировании стала известна дата предполагаемого окончания проекта --- 19.09.2024 г. 
Превышение срока на 19 дней при начальных параметрах планирования.

\chapter{Лабораторная работа №~2}

\section{Тема и цель}

Тема: определение ресурсов и затрат для проекта.

Цель: освоить возможности программы Microsoft Project для определения ресурсов и затрат для проекта.

\section{Тренировочное задание (вариант №~1)}

В ходе выполнения тренировочного задания временной план проекта, подготовленный на предыдущем этапе, был дополнен информацией о ресурсах и определена стоимость проекта. 
На рисунке~\ref{img:screenshot-06} показан заполненный ресурсный лист. 
Так как по условию квалификация рабочих одинаковая, можно добавить один трудовой ресурс и назначить ему 1100\% макс. единиц.
    
\includeimage
    {screenshot-06}
    {f}
    {H}
    {1\textwidth}
    {Ресурсный лист для тренировочного задания}
    
На рисунке~\ref{img:screenshot-07} показана диаграмма Ганта вместе с назначенными ресурсами.
    
\includeimage
    {screenshot-07}
    {f}
    {H}
    {1\textwidth}
    {Диаграмма Ганта для тренировочного задания}
    
На диаграмме можно увидеть, что возникла перегрузка ресурсов. 
Для одновременного выполнения работ C, E и F необходимо 12 рабочих, однако доступно лишь 11.
    
На рисунке~\ref{img:screenshot-08} показан дополненный ресурсный лист --- добавлен ресурс <<Аренда оборудования>>. 
Так как данный ресурс не покупается, а берется в аренду, удобнее назначить ему тип <<Трудовой>>. 
Также для данного ресурса были добавлены затраты на использованиие --- 2 000 руб на установку и наладку арендованного оборудования. 
Календарь --- 24 часа.
    
\includeimage
    {screenshot-08}
    {f}
    {H}
    {1\textwidth}
    {Дополненный ресурсный лист для тренировочного задания}
    
На рисунке~\ref{img:screenshot-09} показано, что аренда оборудования начинается с 3-го дня выполнения задачи B.
    
\includeimage
    {screenshot-09}
    {f}
    {H}
    {1\textwidth}
    {Форма ресурсов для тренировочного задания}
    
Суммарные затраты --- 44 600 руб. 
На рисунке~\ref{img:screenshot-10} показаны суммарные затраты.
    
\includeimage
    {screenshot-10}
    {f}
    {H}
    {1\textwidth}
    {Суммарные затраты для тренировочного задания}
    
В результате выполнения тренировочного задания временной план проекта был дополнен информацией о ресурсах. 
Обнаружена перегрузка ресурсов, а также вычислен бюджет проекта.
    
\section{Основное задание}

\subsection{Задание №~1}

В ходе выполнения первого задания был заполнен ресурсный лист. 
На рисунке~\ref{img:screenshot-11} показан ресурсный лист.
    
\includeimage
    {screenshot-11}
    {f}
    {H}
    {1\textwidth}
    {Ресурсный лист для основного задания}

\subsection{Задание №~2}

В ходе выполнения второго задания ресурсы были назначены задачам. 
На рисунках~\ref{img:screenshot-12-01}--\ref{img:screenshot-12-02} показана диаграмма Ганта.
    
\includeimage
    {screenshot-12-01}
    {f}
    {H}
    {1\textwidth}
    {Диаграмма Ганта для основного задания --- часть 1}
    
\includeimage
    {screenshot-12-02}
    {f}
    {H}
    {1\textwidth}
    {Диаграмма Ганта для основного задания --- часть 2}
    
На рисунке~\ref{img:screenshot-13} показан ресурсный лист, на котором можно увидеть, что возникли перегрузки ресурсов.
    
\includeimage
    {screenshot-13}
    {f}
    {H}
    {1\textwidth}
    {Ресурсный лист с перегрузками для основного задания}
    
Системный аналитик работает одновременно над двумя задачами --- анализ и построение структуры базы объектов, а также анализ и проектирование ядра GIS.

Технический писатель занимается одновременно созданием справочной системы и написание руководства для пользователя.

Художник-дизайнер одновременно работает над разработкой дизайна сайта и разработкой дизайна руководства.

На рисунке~\ref{img:screenshot-14} показаны дополнительные фиксированные затраты.
    
\includeimage
    {screenshot-14}
    {f}
    {H}
    {1\textwidth}
    {Фиксированные затраты для основного задания}
    
На рисунке~\ref{img:screenshot-15} показан дополненный ресурсный лист с арендой сервера.
    
\includeimage
    {screenshot-15}
    {f}
    {H}
    {1\textwidth}
    {Дополненный ресурсный лист для основного задания}
    
Так как это аренда оборудования, удобнее обозначить данный ресурс как трудовой и назначить календарь 24 часа.
    
На рисунке~\ref{img:screenshot-16} показана задача, для которой добавлена аренда сервера.
    
\includeimage
    {screenshot-16}
    {f}
    {H}
    {1\textwidth}
    {Аренда сервера}
    
Стоимость аренды --- 6 210 руб.

На рисунке~\ref{img:screenshot-17-02} показаны общие затраты.
    
\includeimage
    {screenshot-17-02}
    {f}
    {H}
    {1\textwidth}
    {Общие затраты}

\subsection{Задание №~3}

В ходе выполнения третьего задания была получена информация о ресурсах в графическом виде. 
На рисунке~\ref{img:screenshot-17} показан результат структуризации ресурсов по группам.
    
\includeimage
    {screenshot-17}
    {f}
    {H}
    {1\textwidth}
    {Структуризация ресурсов по группам}
    
На рисунке~\ref{img:screenshot-18} показана информация о затратах по ресурсным группам в графическом виде, а на рисунке~\ref{img:screenshot-19} показана информация о трудозатратах по ресурсным группам в графическом виде.

\includeimage
    {screenshot-18}
    {f}
    {H}
    {1\textwidth}
    {Информация о затратах по ресурсным группам в графическом виде}

\includeimage
    {screenshot-19}
    {f}
    {H}
    {1\textwidth}
    {Информация о трудозатратах по ресурсным группам в графическом виде}
    
Исходя из двух круговых диаграмм можно сделать следующие выводы:
\begin{enumerate}
\item[1)] наибольшее количество финансовых вложений требует программирование, притом что оно составляет лишь 29\% от общих трудозатрат;
\item[2)] наибольшие трудозатраты приходятся на аренду оборудования;
\item[3)] ввод данных требует лишь 11\% бюджета при 25\% трудозатрат;
\item[4)] на анализ приходится 2\% трудозатрат и аж 10\% бюджета.
\end{enumerate}

\section{Вывод}

Суммарные затраты --- 48 286 руб. 
Итог --- затраты удовлетворяют бюджету (50 000 руб).

Нужно стремиться сократить трудозатраты на анализ, а также уменьшить затраты на аренду оборудования. 
Как вариант, можно купить оборудование или перейти в облако. 
Приобретение дает возможность использовать оборудование в будущих проектах.

\chapter{Лабораторная работа №~3}

\section{Тема и цель}

Тема: оптимизация параметров проекта, выравнивание загрузки ресурсов, учет периодических задач и минимизация критического пути.

Цель: получить навыки использования программы Microsoft Project для оптимизации временных и финансовых показателей проекта.

\section{Основное задание}

\subsection{Задание №~1}

При выполнении лабораторной работы № 2 появились перегрузки ресурсов.

Системный аналитик работает одновременно над двумя задачами --- анализ и построение структуры базы объектов, а также анализ и проектирование ядра GIS.

Технический писатель занимается одновременно созданием справочной системы и написанием руководства для пользователя.

Художник-дизайнер одновременно работает над разработкой дизайна сайта и разработкой дизайна руководства.

В ходе выполнения первого задания была ликвидирована перегрузка ресурсов в проекте. 
На рисунках~\ref{img:screenshot-20}--\ref{img:screenshot-22} показаны диаграммы Ганта до использования выравнивания, а на рисунках~\ref{img:screenshot-21}--\ref{img:screenshot-23} --- после.
    
\includeimage
    {screenshot-20}
    {f}
    {H}
    {1\textwidth}
    {Диаграмма Ганта с длительностью задач до выравнивания}
    
\includeimage
    {screenshot-22}
    {f}
    {H}
    {1\textwidth}
    {Диаграмма Ганта с затратами до выравнивания}
    
\includeimage
    {screenshot-21}
    {f}
    {H}
    {1\textwidth}
    {Диаграмма Ганта с длительностью задач после выравнивания}
    
\includeimage
    {screenshot-23}
    {f}
    {H}
    {1\textwidth}
    {Диаграмма Ганта с затратами после выравнивания}
    
После выравнивания срок завершения работы над проектом увеличился на 4 дня, а бюджет уменьшился на 144 руб.

Дата заверешения увеличилась, так как увеличилось время на разработку и поддержку веб-сайта. 
Также сдвинулись сроки работы над построением базы объектов и созданием справочной системы. 
Построение базы объектов теперь занимает на 4 дня меньше. 
Вследствие этого бюджет уменьшился, так как временя аренды сервера для построения базы объектов тоже уменьшилось.

\subsection{Задание №~2}

В ходе выполнения второго задания была добавлена повторяющаяся задача <<Совещание>>, которая выполняется раз в неделю по средам. 
К данной задаче привлекаются все ресурсы, кроме программистов № 1--4, наборщиков данных и сервера. 
На рисунках~\ref{img:screenshot-24}--\ref{img:screenshot-25} показана добавленная повторяющаяся задача.
    
\includeimage
    {screenshot-24}
    {f}
    {H}
    {1\textwidth}
    {Сведения о повторяющейся задаче}
    
\includeimage
    {screenshot-25}
    {f}
    {H}
    {1\textwidth}
    {Назначенные повторяющейся задаче ресурсы}
    
Затем выполняется автоматическое выравнивание, после чего можно увидеть обновленные суммарные затраты --- 68 217 руб. 
Они на данный момент первышают бюджет на более чем 18 000 руб. 
Дата окончания проекта --- 27.09.2024 г., что превышает первоначальный срок завершения проекта.
    
\includeimage
    {screenshot-26}
    {f}
    {H}
    {1\textwidth}
    {Диаграмма Ганта}
    
Чтобы уменьшить затраты необходимо добавить план B для всех ресурсов, которые используются на совещаниях. 
В этих планах не учитываются затраты на использование.
    
\includeimage
    {screenshot-27}
    {f}
    {H}
    {1\textwidth}
    {План B для ресурсов}
    
\includeimage
    {screenshot-28}
    {f}
    {H}
    {1\textwidth}
    {Назначение совещаниям плана B}
    
\includeimage
    {screenshot-29}
    {f}
    {H}
    {1\textwidth}
    {Диаграмма Ганта после оптимизации затрат}
    
Затраты уменьшились до 49 947 руб, что удовлетворяет бюджету проекта. 
Однако дата завершения --- 27.09.2024 г., ее нужно уменьшать.
    
\includeimage
    {screenshot-30}
    {f}
    {H}
    {1\textwidth}
    {Дата завершения проекта}

\subsection{Задание №~3}

\includeimage
    {screenshot-31}
    {f}
    {H}
    {1\textwidth}
    {Критический путь с сортировкой задач по длительности}

Из всех задач, лежащих на критическом пути, наибольшую длительность имеют задачи, связанные с программированием. 
То есть они оказывают наибольшее влияние на дату завершения проекта.

Проанализировав назначенные на эти задачи ресурсы, можно прийти к выводу, что программисты распределены неравномерно. 
Можно перераспределить их между задачами, что способствует ускорению завершения этих задач и всего проекта. 
Затем следует удалить все совещания позже даты завершения последней задачи проекта. 
Для равномерного распределения программистов по их задачам были изменены назначения на следующие задачи:
\begin{enumerate}
\item[1)] <<Программирование интерфейса>> --- программисты № 1--4;
\item[2)] <<Программирование средств обработки>> --- программисты № 1--4;
\item[3)] <<Создание модели ядра>> --- программисты № 1--4;
\item[4)] <<Создание рабочей версии ядра>> --- программисты № 1--4;
\item[5)] <<Тестирование сайта>> --- программисты № 1--4.
\end{enumerate}

\includeimage
    {screenshot-33}
    {f}
    {H}
    {1\textwidth}
    {Диаграмма Ганта после перераспределения ресурсов}

После перераспределения программистов на их задачи новых перегрузок не возникло. 
Исходя из результатов лабороторной работы № 2, программисты являются высокооплачиваемыми специалистами, значит, сокращение времени их работы позволит существенно уменьшить затраты. 
Дата завершения проекта теперь 24.07.2024 г., что не превышает первоначального срока, а суммарные затраты на проект уменьшилась --- до 48 302.90 руб, что не превышает бюджета проекта.

На рисунках~\ref{img:screenshot-18}--\ref{img:screenshot-34} показана информация о затратах и трудозатратах по ресурсным группам в графическом виде.

\includeimage
    {screenshot-18}
    {f}
    {H}
    {.9\textwidth}
    {Информация о затратах по ресурсным группам в графическом виде (ЛР № 2)}

\includeimage
    {screenshot-35}
    {f}
    {H}
    {.9\textwidth}
    {Информация о затратах по ресурсным группам в графическом виде (ЛР № 3)}

\includeimage
    {screenshot-19}
    {f}
    {H}
    {.9\textwidth}
    {Информация о трудозатратах по ресурсным группам в графическом виде (ЛР № 2)}

\includeimage
    {screenshot-34}
    {f}
    {H}
    {.9\textwidth}
    {Информация о трудозатратах по ресурсным группам в графическом виде (ЛР № 3)}

После оптимизации соотношение <<трудозатраты/затраты>> действительно изменилось: анализ увеличился с 5 до 5.5, программирование уменьшилось с 1.72 до 1.71, аренда оборудования увеличилась с 0.39 до 0.41, ввод данных уменьшился с 0.44 до 0.42, документация увеличилась с 1 до 1.5.

\includeimage
    {screenshot-36}
    {f}
    {H}
    {.9\textwidth}
    {Сохранение базового плана}

\section{Вывод}

В конце выполнения оптимизации суммарные затраты составили --- 48 318,90 руб, а дата окончания проекта --- 24.07.2024 г. 
Итог --- оба показателя удовлетворяют исходным данным.

\chapter{Лабораторная работа №~4}

\section{Тема и цель}

Тема: актуализация параметров проекта, ввод фактических данных для задач и просмотр отклонений от контрольного плана.

Цель: знакомство с возможностями программы Microsoft Project по контролю за ходом реализации проекта.

\section{Основное задание}

\subsection{Задание №~1}

\includeimage
    {screenshot-37}
    {f}
    {H}
    {.9\textwidth}
    {Дата завершения и затраты по плану}
    
Устанавливается дата отчета --- 16.05.2024 г.

\includeimage
    {screenshot-38}
    {f}
    {H}
    {.4\textwidth}
    {Дата отчета}
    
Далее вносятся фактические данные для отдельных задач проекта. 
Задача №~6 (по заданию №~26, однако это, вероятно, опечатка) фактически завершилась 12.04.2024 г.

\includeimage
    {screenshot-39}
    {f}
    {H}
    {.9\textwidth}
    {Завершение работы}
    
\includeimage
    {screenshot-40}
    {f}
    {H}
    {.9\textwidth}
    {Результат завершения работы после запланированного срока}
    
Срок окончания проекта не поменялся. 
Сдивнулись только даты завершения задач 3, 6 и 17. 
Сроки 6-й были изменены вручную, 3-я --- фаза и зависит от 6-й, а 17-я --- зависит от 3-й.

Программист №~4 уволился 01.04.2024 г. 
В таблице <<Форма ресурсов>> можно увидеть, что все задачи, кроме одной, он должен был выполнять после 1-го апреля, поэтому можно просто удалить этот ресурс у данных задач. 
Для задачи <<Создание модели ядра>> можно указать запланированное окончание 01.04.2024 г.

\includeimage
    {screenshot-41}
    {f}
    {H}
    {.9\textwidth}
    {До удаления ресурса}
    
\includeimage
    {screenshot-42}
    {f}
    {H}
    {.9\textwidth}
    {После удаления ресурса}
    
Очевидно, после таких действий появляются перегрузки других ресурсов. 
Для остальных программистов был увеличен рабочий день до 9-и часов и повышена з/п на 5\%.

\includeimage
    {screenshot-43}
    {f}
    {H}
    {.9\textwidth}
    {Изменение рабочего дня}

\includeimage
    {screenshot-44}
    {f}
    {H}
    {.9\textwidth}
    {Изменение з/п с 1-го апреля}
    
Далее был добавлен стажер с з/п 3 руб/ч, который задействуется в задачах, связанных с тестированием (таких 2-е задачи --- 15-я и 26-я).

\includeimage
    {screenshot-45}
    {f}
    {H}
    {.9\textwidth}
    {Стажер}

\includeimage
    {screenshot-46}
    {f}
    {H}
    {.9\textwidth}
    {Назначение стажера задачам}
    
Далее была куплена лицензия на специализированное ПО за 5 000 руб/год. 
Так как весь проект должен длиться не более 6-и месяцев, будет оплачена годовая цена и затраты на обслуживание --- 400 руб

\includeimage
    {screenshot-47}
    {f}
    {H}
    {.9\textwidth}
    {Лицензия}

\includeimage
    {screenshot-48}
    {f}
    {H}
    {.9\textwidth}
    {Назначение лицензии задаче <<Создание мультимедиа наполнения>>}
    
С 8-го апреля на 10\% была увеличена зарплата мультимедиа-корреспондента.

\includeimage
    {screenshot-49}
    {f}
    {H}
    {.9\textwidth}
    {Увеличение з/п мультимедиа-корреспондента}
    
С 01.04.2024 г. в совещаниях стали участвовать только те специалисты, чьи задачи реализуются на текущей неделе. 
От группы программистов участвует по-прежнему ведущий программист.

\includeimage
    {screenshot-50}
    {f}
    {H}
    {.9\textwidth}
    {Измененные ресурсы для совещаний}
    
С 8-го апреля заключили договор с партнерской организацией, что позволило отказаться от арендованного сервера, так как организация предоставила свой сервер на безвозмездной основе.

\includeimage
    {screenshot-51}
    {f}
    {H}
    {.9\textwidth}
    {Уменьшение времени использования сервера}

Помимо этого, договор предусматривает оказание организацией-партнером сервисных услуг на сумму 3200 рублей до конца проекта. 
Данные затраты можно оформить как фиксированные.

\includeimage
    {screenshot-52}
    {f}
    {H}
    {.4\textwidth}
    {Плата за услуги}
    
После выполнения автоматического выравнивания для устранения перегрузок ресурсов можно увидеть, что срок окончания проекта сдвинулся на 2.5 недели, но все еще меньше 6-и месяцев. 
    
\includeimage
    {screenshot-54}
    {f}
    {H}
    {.9\textwidth}
    {Дата окончания проекта и затраты после внесенных изменений}
    
А вот затраты превысили бюджет. 
Один из вариантов понижения затрат --- привлечь стажера к выполнению двух задач.

\includeimage
    {screenshot-58}
    {f}
    {H}
    {.9\textwidth}
    {Дата окончания проекта и затраты после добавления стажера задачам 10 и 16}
    
\includeimage
    {screenshot-56}
    {f}
    {H}
    {.9\textwidth}
    {Задачи, выполненые до даты отчета}
    
\includeimage
    {screenshot-57}
    {f}
    {H}
    {.9\textwidth}
    {Линия прогресса}
    
\section{Вывод}

В результате актуализации параметров проекта и устранения временных и финансовых отклонений были получены затраты --- 45 267,32 руб --- и дата окончания проекта --- 15.08.2024 г. 
Итог --- оба показателя удовлетворяют исходным данным.

\chapter{Лабораторная работа №~5}

\section{Тема и цель}

Тема: контроль хода выполнения проекта с помощью средств анализа затрат; работа с отчетами.

Цель: освоение возможностей программы Microsoft Project по управлению финансовыми потоками на основе анализа затрат.

\section{Основное задание}

\subsection{Задание №~1}

В результате выполнения ЛР-4 получены затраты в размере 45 267,32 руб и дата окончания проекта 15.08.2024 г.

\includeimage
    {screenshot-58}
    {f}
    {H}
    {.9\textwidth}
    {Дата окончания и затраты по плану}
    
\includeimage
    {screenshot-38}
    {f}
    {H}
    {.4\textwidth}
    {Дата отчета}
    
\includeimage
    {screenshot-62}
    {f}
    {H}
    {.9\textwidth}
    {Прямые и косвенные затраты}
    
На 16.05.2024 г. выполнены задачи №~1--17, кроме 7, 11 и 16. 
Прямые затраты (связанные с выполнением работ): 
\begin{enumerate}
\item[1)] 3 200 руб за услуги от компании-партнера;
\item[2)] 1 000 руб фиксированных затрат для задач 2, 8 и 12. 
\end{enumerate}
Косвенные затраты (связанные с использованием ресурсов): 
\begin{enumerate}
\item[1)] 4 542,25 руб для задачи 2;
\item[2)] 9 158,36 руб для задачи 8;
\item[3)] 16 688,87 руб для задачи 12. 
\end{enumerate}

Итого: 4 200 руб прямых и 30 389,48 руб косвенных затрат по состоянию на 16.05.2024 г.

Используя таблицу освоенного объема можно увидеть, что индекс отклонения от календарного плана меньше 1, а индекс отклонения по стоимости --- больше 1. 
Это значит, что выполненный объем работ по состоянию на 16.05.2024 г. меньше, чем запланированный, а вот затрат меньше, чем по плану. 
То есть, можно использовать некоторое количество финансов для увеличения скорости выполнения задач.
    
\includeimage
    {screenshot-59}
    {f}
    {H}
    {.9\textwidth}
    {Таблица освоенного объема --- часть 1}
    
\includeimage
    {screenshot-60}
    {f}
    {H}
    {.9\textwidth}
    {Таблица освоенного объема --- часть 2}
    
\includeimage
    {screenshot-61}
    {f}
    {H}
    {.9\textwidth}
    {Таблица освоенного объема --- часть 3}
    
Предварительная оценка по завершении меньше 50 000 руб, а предполагаемая дата окончания --- 15.08.2024 г., поэтому несмотря на то, что на дату отчета выполнение проекта идет медленнее, чем ожидалось, проект должен выполниться вовремя.
    
\subsection{Задание №~2}
    
\includeimage
    {screenshot-63}
    {f}
    {H}
    {.9\textwidth}
    {Отчет по затратам}
    
Руководитель проекта будет испытывать наибольшую потребность в деньгах во время выполнения задачи <<Создание ядра GIS>> (на гистограмме видно, что это примерно 18 000 руб).

\includeimage
    {screenshot-64}
    {f}
    {H}
    {.9\textwidth}
    {Отчет по превышению затрат}
    
Бюджетную стоимость превышают задачи <<Начало проекта>> и <<Создание мультимедиа-наполнения>>. 
Притом <<Начало проекта>> превышает бюджет на 3 200 руб из-за того, что было подписано соглашение с компанией-партнером на предоставление услуг. 
За счет этих затрат экономятся средства, выделенные на аренду сервера.

\subsection{Задание №~3}

\includeimage
    {screenshot-65}
    {f}
    {H}
    {1\textwidth}
    {Дата окончания и затраты по результатам выполнения ЛР-2}
    
Уберем все значения в столбце <<Предшественники>>.
    
\includeimage
    {screenshot-66}
    {f}
    {H}
    {1\textwidth}
    {Дата окончания и затраты после сброса параметра <<Предшественники>>}
    
Каскадная модель --- модель процесса разработки программного обеспечения, в которой процесс разработки выглядит как поток, последовательно проходящий фазы анализа требований, проектирования, реализации, тестирования, интеграции и поддержки. 
Поставим параметр <<Предшественники>> так, чтобы план соответствовал этой модели.
    
\includeimage
    {screenshot-67}
    {f}
    {H}
    {1\textwidth}
    {Дата окончания и затраты после перехода на каскадную модель}
    
Задачи <<Создание интерфейса>>, <<Построение базы объектов>> и <<Создание ядра GIS>> могут выполняться одновременно, так как не зависят друг от друга. 
Такая же ситуация и с задачами <<Создание мультимедиа-наполнения>> и <<Создание справочной системы>>. 
Все остальные задачи должны выполняться друг за другом в соответствии с каскадной моделью.
    
\includeimage
    {screenshot-68}
    {f}
    {H}
    {1\textwidth}
    {Дата окончания и затраты после внесения изменений в задачах на более низком уровне --- часть 1}
    
\includeimage
    {screenshot-69}
    {f}
    {H}
    {1\textwidth}
    {Дата окончания и затраты после внесения изменений в задачах на более низком уровне --- часть 2}
   
Так как дата окончания проекта позже, чем требует того ТЗ, необходимо внести некоторые изменения.
    
Пусть тестирование начнется вместе в этапом наполнения сайта. 
Тесты могут проводиться с использованием заглушек.    
    
\includeimage
    {screenshot-70}
    {f}
    {H}
    {1\textwidth}
    {Дата окончания и затраты после внесения изменений в задачу <<Тестирование сайта>>}
    
Пусть создание рабочей версии ядра выполнения одновременно с тестирование модели ядра.
    
\includeimage
    {screenshot-71}
    {f}
    {H}
    {1\textwidth}
    {Дата окончания и затраты после внесения изменений в задачу <<Создание рабочей версии ядра>>}
    
\section{Вывод}

Итог --- дата окончания 27.08.2024 г. и затраты 48 286,00 руб. 
Эти показатели удовлетворяют условию. 
Однако при каскадной модели дата окончания уменьшилась на 3 недели, а бюджет остался таким же.

Также появились перегрузки ресурсов. 
Системный аналитик занят анализом и построение структуры базы и анализом и проектированием ядра. 
Программист №~1 занят тестированием ядра и созданием его рабочей версии. 
Художник-дизайнер занимается одновременно разработкой дизайна руководства и сайта.

\end{document}
