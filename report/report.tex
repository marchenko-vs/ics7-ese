\documentclass{bmstu}

\usepackage{biblatex}
\usepackage{array}
\usepackage{amsmath}

\addbibresource{inc/biblio/sources.bib}

\begin{document}

\makereporttitle
    {Информатика и системы управления}
    {Программное обеспечение ЭВМ и информационные технологии}
    {лабораторным работам №~1--5}
    {Экономика программной инженерии}
    {Планирование программного проекта в Microsoft Project}
    {}
    {Марченко~В./ИУ7-83Б}
    {Барышникова~М.~Ю.}

{\centering \maketableofcontents}

\chapter{Лабораторная работа №~1}

\section{Цель и задачи}

Тема: планирование программного проекта в Microsoft Project --- настройка рабочей среды и создание нового проекта.

Цель: освоить возможности программы Microsoft Project для планирования проекта по разработке программного обеспечения.

Задачи: выполнить тренировочное задание для варианта №~1 и основное задание на лабораторную работу.

\section{Тренировочное задание (вариант №~1)}

В ходе выполнения тренировочного задания было осуществлено планирование проекта с учетом определенных связей между задачами. 
Сначала была настроена рабочая среда проекта. 
Единица измерения длительности выполения задач --- дни, 8-часовой рабочий день с 9:00 до 18:00, 5 рабочих дней в неделю, 40 часов в неделю. 
Были учтены праздничные дни --- 8 Марта, День Победы и День международной солидарности трудящихся. 
Дата начала проекта --- 1 марта 2024 г., дата окончания --- 8 мая 2024 г. 
На рисунке~\ref{img:screenshot-01} показана диаграмма Ганта с включенной суммарной задачей.
    
\includeimage
    {screenshot-01}
    {f}
    {H}
    {1\textwidth}
    {Диаграмма Ганта для тренировочного задания №~1}
    
По умолчанию при планировании проекта Microsoft Project использует фиксированный объем ресурсов в качестве типа задач.
    
\section{Основное задание}

Содержание проекта. 
Команда разработчиков из 16-и человек занимается созданием карты города на основе собственного модуля отображения. 
Проект должен быть завершен в течение 6-и месяцев. 
Бюджет проекта: 50 000 рублей.

\subsection{Задание №~1}

В ходе выполнения первого задания была осуществлена настройка рабочей среды проекта с помощью раздела <<Расписание>> окна <<Параметры Project>>. 
На рисунке~\ref{img:screenshot-02} показаны параметры планирования.

Единица измерения длительности выполения задач --- недели, 8-часовой рабочий день с 9:00 до 18:00, 5 рабочих дней в неделю, 40 часов в неделю. 
Дата начала проекта --- 1 марта 2024 г.
    
\includeimage
    {screenshot-02}
    {f}
    {H}
    {1\textwidth}
    {Параметры планирования для основного задания}
    
Учтены праздничные дни в календаре вплоть до августа 2024 г. 
На экран выведена суммарная задачу проекта и заполнена вкладка <<Заметки>> информацией об основных параметрах проекта.

\subsection{Задание №~2}

В ходе выполнения второго задания был создан список задач. 
На рисунках~\ref{img:screenshot-03-01}--\ref{img:screenshot-03-02} показан список задач.
    
\includeimage
    {screenshot-03-01}
    {f}
    {H}
    {1\textwidth}
    {Список задач --- часть 1}
    
\includeimage
    {screenshot-03-02}
    {f}
    {H}
    {1\textwidth}
    {Список задач --- часть 2}
    
Задачи 1 и 27 являются задачами вехами, поэтому они имеют нулевую
продолжительность. 
Задачи 2, 3, 8, 12, 19 и 22 в задании №~3 будут преобразованы в фазы проекта, поэтому их длительность условная.

\subsection{Задание №~3}

В ходе выполнения третьего задания список задач был структурирован с помощью кнопки <<Понизить уровень задачи>> на вкладке <<Задача>>. 
На рисунках~\ref{img:screenshot-04-01}--\ref{img:screenshot-04-02} показан структурированный список задач.
    
\includeimage
    {screenshot-04-01}
    {f}
    {H}
    {1\textwidth}
    {Структурированный список задач --- часть 1}
    
\includeimage
    {screenshot-04-02}
    {f}
    {H}
    {1\textwidth}
    {Структурированный список задач --- часть 2}

\subsection{Задание №~4}

В ходе выполнения четвертого задания были установлены связи разных типов между задачами. 
На рисунках~\ref{img:screenshot-05-01}--\ref{img:screenshot-05-02} показан результат выполнения основного задания --- план проекта по разработке программного обеспечения.
    
\includeimage
    {screenshot-05-01}
    {f}
    {H}
    {1\textwidth}
    {План проекта по разработке программного обеспечения --- часть 1}
    
\includeimage
    {screenshot-05-02}
    {f}
    {H}
    {1\textwidth}
    {План проекта по разработке программного обеспечения --- часть 2}
    
\section{Вывод}

В ходе выполнения лабораторной работы №~1 была достигнута поставленная цель --- освоены возможности программы Microsoft Project для планирования проекта по разработке программного обеспечения. 
Для закрепления полученных навыков были выполнены тренировочное и основное задания.

По ТЗ проект должен был выполнится за 6 месяцев, но при планировании стало известна дата предполагаемого окончания разработки --- 19 сентября 2024 г. 
Превышение срока на 19 дней при начальных параметрах планирования.

Получены следующие навыки:
\begin{enumerate}
\item[1)] ввод информации о задачах;
\item[2)] добавление дополнительных сведений о задачах;
\item[3)] указание продолжительности задач;
\item[4)] создание иерархической структуры работ проекта:
\item[5)] создание нескольких уровней вложения;
\item[6)] сворачивание и разворачивание списка задач;
\item[7)] связывание задач проекта;
\item[8)] корректировка типа связи;
\item[9)] задание времени задержки и опережения;
\item[10)] установление временных ограничений;
\item[11)] изменение типов ограничений;
\item[12)] удаление ограничений;
\item[13)] игнорирование ограничений;
\item[14)] задание предельного срока завершения работ (deadline).
\end{enumerate}

\chapter{Лабораторная работа №~2}

\section{Цель и задачи}

Тема: определение ресурсов и затрат для проекта.

Цель: освоить возможности программы Microsoft Project для определения ресурсов и затрат для проекта.

Задачи: выполнить тренировочное задание для варианта №~1 и основное задание на лабораторную работу.

\section{Тренировочное задание (вариант №~1)}

В ходе выполнения тренировочного задания временной план проекта, подготовленный на предыдущем этапе, был дополнен информацией о ресурсах и опредена стоимость проекта. 
На рисунке~\ref{img:screenshot-06} показан заполненный ресурсный лист.
    
\includeimage
    {screenshot-06}
    {f}
    {H}
    {1\textwidth}
    {Ресурсный лист для тренировочного задания №~1}
    
На рисунке~\ref{img:screenshot-07} показана диаграмма Ганта вместе с назначенными ресурсами.
    
\includeimage
    {screenshot-07}
    {f}
    {H}
    {1\textwidth}
    {Диаграмма Ганта для тренировочного задания №~1}
    
На рисунке~\ref{img:screenshot-08} показан дополненный ресурсный лист --- добавлен ресурс <<Аренда оборудования>>.
    
\includeimage
    {screenshot-08}
    {f}
    {H}
    {1\textwidth}
    {Ресурсный лист для тренировочного задания №~1}
    
На рисунке~\ref{img:screenshot-09} показано, что аренда оборудования начинается с 3-го дня выполнения задачи B.
    
\includeimage
    {screenshot-09}
    {f}
    {H}
    {1\textwidth}
    {Форма ресурсов для тренировочного задания №~1}
    
Также была добавлены фиксированные затраты --- 3 000 руб на установку и наладку арендованного оборудования. 
Суммарные затраты --- 274 840 руб. 
На рисунке~\ref{img:screenshot-10} показаны суммарные затраты.
    
\includeimage
    {screenshot-10}
    {f}
    {H}
    {1\textwidth}
    {Суммарные затраты для тренировочного задания №~1}
    
\section{Основное задание}

\subsection{Задание №~1}

В ходе выполнения первого задания был заполнен ресурсный лист. 
На рисунке~\ref{img:screenshot-11} показан ресурсный лист.
    
\includeimage
    {screenshot-11}
    {f}
    {H}
    {1\textwidth}
    {Ресурсный лист для основного задания}

\subsection{Задание №~2}

В ходе выполнения второго задания ресурсы были назначены задачам. 
На рисунках~\ref{img:screenshot-12-01}--\ref{img:screenshot-12-02} показана диаграмма Ганта.
    
\includeimage
    {screenshot-12-01}
    {f}
    {H}
    {1\textwidth}
    {Диаграмма Ганта для основного задания --- часть 1}
    
\includeimage
    {screenshot-12-02}
    {f}
    {H}
    {1\textwidth}
    {Диаграмма Ганта для основного задания --- часть 2}
    
На рисунке~\ref{img:screenshot-13} показан ресурсный лист, на котором можно увидеть, что возникли перегрузки ресурсов.
    
\includeimage
    {screenshot-13}
    {f}
    {H}
    {1\textwidth}
    {Ресурсный лист с перегрузками для основного задания}
    
Системный аналитик работает одновременно над двумя задачами --- анализ и построение стурктуры базы объектов, а также анализ и проектирование ядра GIS.

Технический писатель занимается одновременно созданием справочной системы и написание руководства для пользователя.

Художник-дизайнер одновременно работает над разработкой дизайна сайта и разработкой дизайна руководства.

На рисунке~\ref{img:screenshot-14} показаны дополнительные фиксированные затраты.
    
\includeimage
    {screenshot-14}
    {f}
    {H}
    {1\textwidth}
    {Фиксированные затраты для основного задания}
    
На рисунке~\ref{img:screenshot-15} показан дополненный ресурсный лист с арендой сервера.
    
\includeimage
    {screenshot-15}
    {f}
    {H}
    {1\textwidth}
    {Дополненный ресурсный лист для основного задания}
    
На рисунке~\ref{img:screenshot-16} показана задача, для которой добавлена аренда сервера.
    
\includeimage
    {screenshot-16}
    {f}
    {H}
    {1\textwidth}
    {Аренда сервера}

\subsection{Задание №~3}

В ходе выполнения третьего задания список задач была получена информация о ресурсах в графическом виде. 
На рисунке~\ref{img:screenshot-17} показана информация о ресурсах в графическом виде.
    
\includeimage
    {screenshot-17}
    {f}
    {H}
    {1\textwidth}
    {Информация о ресурсах в графическом виде}
    
На рисунке~\ref{img:screenshot-18} показана информация о затратах ресурсов в графическом виде.
    
\includeimage
    {screenshot-18}
    {f}
    {H}
    {1\textwidth}
    {Информация о затратах ресурсов в графическом виде}

\section{Вывод}

В ходе выполнения лабораторной работы №~2 была достигнута поставленная цель --- освоены возможности программы Microsoft Project для определения ресурсов и затрат для проекта. 
Для закрепления полученных навыков были выполнены тренировочное и основное задания.

Суммарные затраты --- 43 436 руб. 
Итог --- затраты удовлетворяют бюджету (50 000 руб).

\end{document}
