\documentclass{bmstu}

\usepackage{biblatex}
\usepackage{array}
\usepackage{amsmath}

\addbibresource{inc/biblio/sources.bib}

\begin{document}

\makereporttitle
    {Информатика и системы управления}
    {Программное обеспечение ЭВМ и информационные технологии}
    {лабораторной работе №~6}
    {Экономика программной инженерии}
    {Предварительная оценка параметров программного проекта}
    {2}
    {Марченко~В./ИУ7-83Б}
    {Барышникова~М.~Ю.}

{\centering \maketableofcontents}

\chapter{Лабораторная работа}

\section{Цель работы}

Целью лабораторной работы является ознакомление с существующими методиками предварительной оценки параметров программного проекта и практическая оценка затрат на примере методики COCOMO (COnstructive COst MOdel --- конструктивная модель стоимости).

\section{Методика COCOMO}

COnstructive COst MOdel --- это алгоритмическая модель оценки стоимости разработки программного обеспечения, разработанная Барри Боэмом. 
Модель использует простую формулу регрессии с параметрами, определенными из данных, собранных по ряду проектов.

Базовый уровень рассчитывает трудоемкость и стоимость разработки как функцию от размера программы. Размер выражается в оценочных тысячах строк кода (KLOC --- kilo lines of code).

Есть три режима...
    
\section{Задание №~1}

Исследовать степень влияния различных драйверов затрат на трудоемкость (РМ) и время разработки (ТМ) для промежуточной модели COCOMO. 
Для этого проанализировать, как меняется трудоемкость и время выполнения проекта при различных уровнях автоматизации среды (драйверы MODP --- использование современных методов и TOOL --- использование программных инструментов) и разном уровне способностей ключевых членов команды (драйверы ACAP --- способности аналитика, PCAP --- способности программиста). 
Взять за основу промежуточный тип проекта и при фиксированном значении размера программного кода (SIZE) получить значения PM и ТМ, изменяя значения указанных драйверов от очень низких до очень высоких. 
Результаты исследований оформить графически и сделать соответствующие выводы. 
При необходимости сократить срок выполнения проекта, что повлияет больше: способности персонала или параметры среды? 
При высоком уровне автоматизации (оба драйвера MODP и TOOL высокие) что окажет большее влияние на трудоемкость и время выполнения: высокая надежность (параметр RELY повышается от номинального до высокого) или требование заказчика, чтобы не менее 70\% компонентов разрабатываемого ПО могло использоваться в режиме реального времени (драйвер TIME повышается от номинального до высокого)?

Для начала проварьируем поочередно 4 драйвера (MODP, TOOL, ACAP и PCAP) и получим время и трудозатраты с использованием реализованного ПО.

\includeimage
    {screenshot-02}
    {f}
    {H}
    {1\textwidth}
    {Зависимость времени и трудозатрат от MODP, TOOL, ACAP и PCAP}
    
Из графиков видно, что и время, и трудозатраты для реализации проекта уменьшаются при увеличении коэффициентов автоматизации среды и способностей ключевых членов команды.

Теперь выставим номинальный уровень для всех показателей. 

\includeimage
    {screenshot-03}
    {f}
    {H}
    {1\textwidth}
    {Время и трудозатраты для номинальных уровней}
    
Время --- 24 месяца, трудозатраты --- 288 человеко-месяцев.

Повысим параметры среды до очень высокого уровня. 

\includeimage
    {screenshot-04}
    {f}
    {H}
    {1\textwidth}
    {Время и трудозатраты при очень высоком уровне MODP и TOOL}
    
Время --- 21 месяц, трудозатраты --- 194 человеко-месяца.

Повысим способности персонала до очень высокого уровня. 

\includeimage
    {screenshot-05}
    {f}
    {H}
    {1\textwidth}
    {Время и трудозатраты при очень высоком уровне ACAP и PCAP}
    
Время --- 19 месяцев, трудозатраты --- 143 человеко-месяца.

То есть, если нужно сократить сроки выполнения проекта, нужно учесть, что способности ключевых членов команды будут иметь большее влияние.

Поставим высокий уровень автоматизации. 

\includeimage
    {screenshot-06}
    {f}
    {H}
    {1\textwidth}
    {Время и трудозатраты при высоком уровне MODP и TOOL}
    
Время --- 22 месяца, трудозатраты --- 239 человеко-месяцев.

Поставим высокий уровень RELY.

\includeimage
    {screenshot-07}
    {f}
    {H}
    {1\textwidth}
    {Время и трудозатраты при высоком уровне MODP, TOOL и RELY}
    
Время --- 24 месяца, трудозатраты --- 274 человеко-месяца.

Поставим высокий уровень TIME.

\includeimage
    {screenshot-08}
    {f}
    {H}
    {1\textwidth}
    {Время и трудозатраты при высоком уровне MODP, TOOL и TIME}
    
Время --- 23 месяца, трудозатраты --- 265 человеко-месяца.

Таким образом, при повышении обоих драйверов, время и трудозатраты увеличиваются, но драйвер RELY оказывает большее влияние. 

\section{Задание №~2}

При разработке программного продукта его размер оценивается
примерно в 55 KLOC. 
Этот проект будет представлять собой Web-систему, снабженную устойчивой серверной базой данных. 
Предполагается применение промежуточного варианта. 
Проект предполагает создание продукта средней сложности с номинальными
требованиями по надежности, но с расширенной базой данных. 
Квалификация персонала средняя, однако способности аналитика высокие. 
Оценить параметры проекта.

Режим модели --- промежуточный. 
Показатели размера базы данных и способностей аналитика --- высокие. 
Все остальные --- номинальные. 
Количество строк в тысячах (KLOC) --- 55. 
С помощью реализованного ПО рассчитаем параметры проекта.

\includeimage
    {screenshot-01}
    {f}
    {H}
    {1\textwidth}
    {Оценка параметров проекта}
    
Трудозатраты --- 268 человеко-месяцев, время --- 23 месяца (с учетом планирования). 
Примерный бюджет с учетом средней з/п рабочего 60 тыс. руб --- 16 063 000 руб.

Распределение рабочих по этапам разработки:
\begin{enumerate}
\item[1)] планирование и определение требований --- 3 человека;
\item[2)] проектирование продукта --- 7 человек;
\item[3)] детальное проектирование --- 20 человек;
\item[4)] кодирование и тестирование --- 21 человек;
\item[5)] интеграция и тестирование --- 16 человек.
\end{enumerate}

Наиболее длинные этапы по времени --- планирование и определение требований и проектирование продукта. 
Наиболее трудозатратный --- интеграция и тестирование.

По модели WBS наиболее затратным по финансам является этап программирования.
    
\section{Вывод}

Для задания №~2 были оценены параметры проекта по модели COCOMO: трудозатраты --- 268 человеко-месяцев, время --- 23 месяца (с учетом планирования), примерный бюджет с учетом средней з/п рабочего 60 тыс. руб --- 16 063 000 руб.

Данную оценку можно использовать в качестве начальных данных, однако, сделав выводы из предыдущих 5-и лабораторных работ --- проекты практически никогда не выполняются в сроки. 
То есть нужно иметь в виду, что оцененные параметры очень легко могут измениться, притом в большую сторону.

\end{document}
